% !TeX encoding = UTF-8
% !TeX spellcheck = pl_PL

\def\filename{Rozdział 1}

\chapter{Jednomandatowe Okręgi Wyborcze w~polskich realiach}

\section{Historia ustroju demokratycznego}
\begin{itemize}

\item{\textbf{Starożytna Grecja}} 

Pierwszą rozwiniętą demokracją w~kontekście cywilizacji europejskiej były greckie polis. Sam termin demokracja ma etymologię grecką - \textit{dēmokratiā} - gdzie z~greki \textit{dēmos} to lud, a~\textit{krátos} to władza. Sztandarowym przykładem były tutaj Ateny. Demokracja tzw. ateńska, czyli bezpośrednia, to taka gdzie o ważnych decyzjach decydowała większość uprawnionych do głosowania i~zebranych na agorze obywateli. De facto tylko około 30\% mieszkańców Aten miało status obywatela, z~tego miana wykluczeni byli kobiety, dzieci, niewolnicy oraz imigranci. Pomimo, że w~zgromadzeniach demokratycznych mogli brać udział wszyscy obywatele, to faktycznie władza była sprawowana przez jeszcze węższe grono arystokracji lub oligarchii \cite{Karolczuk}.

\item{\textbf{Starożytny Rzym}}

Późniejszy rozwój demokracji związany był już z~jej odmianą przedstawicielską, taki rozwój był uwarunkowany m.in. rozrastającymi się granicami państw oraz centralizacją prawa. Pierwszym znaczącym przykładem może być tutaj Republika Rzymska istniejąca w~latach 509--27 p.n.e. W~rzeczywistości było to państwo oligarchiczne, gdzie do senatu praktycznie mogli wejść tylko przedstawiciele patrycjatu. W~Rzymie również znaczna większość mieszkańców nie była obywatelami, tak więc nie mieli prawa głosu. Senat rządził Rzymem aż do upadku republiki i~przejęcie władzy monarchistycznej przez pierwszego cesarza Oktawiana Augusta. Od tego momentu monarchia stała się na wiele wieków standardem sprawowania władzy w~cywilizacji łacińskiej.

\item{\textbf{Średniowiecze}}

Aż do XVIII wieku tylko nieliczne państwa miały elementy demokracji. We wszystkich z~nich miała ona charakter elitarny i~oligarchiczny reprezentując interesy wyłącznie stanów arystokratycznych lub mieszczańskich. W~niektórych z~nich uprawnieni mogli wybierać tylko przedstawicieli do parlamentu przy czym monarchia była nadal dziedziczona (Anglia po Magna Carta Libertatum - 1215 r.), w~innych uprawnieni mogli wybierać monarchę (przedchrześcijańskie nordyckie tingi, demokracja szlachecka w~Rzeczypospolitej Obojga Narodów). Najdłużej funkcjonującą republiką w~Europie, i~również na świecie, była Republika Wenecka, gdzie w~oligarchicznej demokracji kupcy wybierali dożę na czas dożywotni.

\item{\textbf{Nowożytność}}

Nowoczesne spojrzenie na demokracje przyniosły zmiany społeczne końca XVIII wieku, gdy na skutek kryzysów ekonomicznych oraz centralizacji absolutystycznych władz monarszych zaczęło dochodzić do powstawania ruchów rewolucyjnych i~wystąpień niższych warstw społecznych. Wspólnym elementem nowo powstałych demokracji były konstytucje. Pierwsza uchwalona przez Korsykę (1769 r.) dawała już prawa wyborcze kobietom i~ustanawiała kadencyjną Radę Wykonawczą. Najbardziej znacząca i~wprowadzającą zupełnie nowe spojrzenie na demokrację jest konstytucja amerykańska (1789 r.). Wprowadziła dwuizbowy parlament oraz kadencyjny urząd prezydenta wybieranego przez reprezentującego ogół uprawnionych do głosowania obywateli Kolegium Elektorów oraz impeachment - sposób jego odwołania.

\item{\textbf{Kryzys demokracji}}

Wiek XIX, a~następnie XX oprócz rozpowszechnienia demokracji na prawie wszystkie państwa europejskie doprowadził również do nowych patologii związanych z~tym ustrojem. Zaczęło dochodzić do coraz większych manipulacji na wyborcach oraz na samych procedurach wyborczych i~prawie z~nimi związanym. Przykładem są na przykład praktycznie monopartyjne Rzesza Niemiecka czy Związek Socjalistycznych Republik Radzieckich, gdzie pomimo organizacji wyborów i~istnienia demokracji pośredniej, obywatele mogli de facto wybierać tylko kandydatów ustalonych przez partie rządzące, lub wyniki wyborów były wręcz fałszowane. Kryzys demokracji doprowadził do wybuchu II Wojny Światowej. Po jej zakończeniu Zachód wprowadzał zasady i~prawa chroniące przed różnego typu manipulacjami, albo w~sposób systemowy uniemożliwiał objęcie władzy przez ugrupowanie totalitarne lub skrajne (zakaz propagowania nazizmu w~RFN, gaullistowska większościowa ordynacja wyborcza promująca partie bardziej centrowe).

\item{\textbf{Współczesność}}

Zmiany społeczne i~kryzys w~bloku komunistycznym w~latach .80 i~.90 doprowadził do upadku ustrojów komunistycznych tzw. dyktatury proletariatu lub demokracji ludowej wprowadzając na ich miejsce w~Europie Wschodniej nowe ustroje demokratyczne oparte na wzorcach zachodnich \cite{Górski}.
\end{itemize}

\section{JOW a~Wielomandatowe Okręgi Wyborcze}

Oprócz tego, że państwa na świecie dzielimy na niedemokratyczne (monarchie, dyktatury, itp.) oraz demokracje, to państwa demokratyczne możemy również podzielić w~zależności od posiadanego systemu ordynacji wyborczej do ich parlamentów. Wyróżnia się dwa główne systemy: proporcjonalny i~większościowy. Ich nazwy odzwierciedlają założenia ich wprowadzenia w~danym państwie. System proporcjonalny daje możliwość reprezentacji swoich poglądów i~interesów każdej grupie społecznej w~skali ogólnokrajowej, której opcja polityczna zdobędzie taką liczbę głosów by wejść do parlamentu. Najważniejszym elementem są tutaj wielomandatowe okręgi wyborcze, w~których mandat mogą uzyskać tylko startujący na liście swojego komitetu w~danym okręgu, jednakże o proporcjach mandatów dla konkretnych komitetów decyduje jego procentowy wynik w~skali całego państwa (wszystkich okręgów łącznie). Dokładny podział mandatów jest rozdysponowany za pomocą metod, z~których najpopularniejsze to: Sainte-Laguë, Hare’a-Niemeyera oraz stosowana w~wyborach do sejmu RP metoda d'Hondta. Stosowanie metody d'Hondta dyskryminuje mniej popularne partie a~faworyzuje partie bardziej popularne. Możliwe jest nawet, że partia która zdobyła mniej niż 50\% głosów zdobyła samodzielną większość parlamentarną (casus Prawa i~Sprawiedliwości w~wyborach do Sejmu RP w~2015 roku).

System większościowy, w~którym kandydaci wybierani są w~Jednomandatowych Okręgach Wyborczych, skupia się bardziej na reprezentowaniu grup obywateli nie według ich poglądów, ale według miejsca ich zamieszkania. Najczęściej wymienianym jego atutem jest rzeczywiste reprezentowanie przez posła ludności ze swojego okręgu wyborczego. Wadą jest to, że duża część mieszkańców po wybraniu kandydata większości, traci jakąkolwiek formę reprezentacji w~parlamencie. Należy natomiast pamiętać, że podobne zjawisko występuje w~systemie proporcjonalnym w~przypadku występowania tzw. progu wyborczego. W~takim wypadku możliwość reprezentacji tracą wyborcy tych partii, które progu nie przekroczyły. De facto funkcjonowanie JOW prowadzi do dominacji dwóch partii politycznych w~danym kraju. Jednak partie w~takich państwach reprezentują większy zakres poglądów i~zupełnie dopuszczalne jest to, by kandydat nie podzielał w~pełni programu swojej partii \cite{Urbańczyk}.

\section{Przykłady implementacji JOW na świecie}
Można zauważyć, że Jednomandatowe Okręgi Wyborcze są przede wszystkim obecne w~państwach z~dominującą kulturą anglo-saską, będących niegdyś częścią Imperium Brytyjskiego. Tam też występują one w~najprostszej formie, oczywiście pewne ich wariancje są również obecne w~wielu innych państwach świata.
\subsection{USA}
Kongres Stanów Zjednoczonych jest podzielony na dwie izby  - Senat i~Izbę Reprezentantów. Do Senatu każdy stan powołuje po dwóch senatorów na 6-letnią kadencję, przy czym wybory odbywają się co 2 lata, tak że zwykle 1/3 mandatów jest obsadzana, a~wybory odbywają się w~2/3 wszystkich stanów. Przy tych wyborach każdy stan jest okręgiem wyborczym, ale mającym do dyspozycji dwa mandaty. Taki system dyskryminuje mieszkańców najludniejszych stanów, ponieważ ich głos jest ma mniejszą wagę niż głos mieszkańców stanów mniejszych. Dla przykładu największa Kalifornia mająca 39 536 653 mieszkańców ma tyle samo senatorów co najmniejsze Wyoming z~578 759 mieszkańcami.

Druga izba Kongresu - Izba Reprezentantów składa się z~435 reprezentantów. Każdy z~nich jest wybierany w~jednomandatowym okręgu wyborczym zwanym \textit{congressional district}. Konstytucja USA zakłada, że jeden reprezentant powinien przypadać na nie mniej niż 30000 obywateli (co obecnie uważa się za przepis archaiczny). Każdy stan po każdym spisie powszechnym dzielony jest na okręgi proporcjonalnie do liczby jego mieszkańców, przy czym każdy powinien mieć przynajmniej jednego reprezentanta niezależnie od jego liczby mieszkańców. Kadencja trwa 2 lata. Obecnie 7 stanów ma prawo wyboru tylko jednego reprezentanta, zaś najludniejszy stan - Kalifornia - ma prawo wyboru aż 53 reprezentantów. 

Obywatele USA, którzy nie mieszkają na terenie jakiegokolwiek stanu są pozbawieni prawa wyboru członków Kongresu, tj. Dystrykt Kolumbii, Portoryko oraz inne terytoria USA. 

Poza nielicznymi wyjątkami (m.in. Georgia) do zdobycia mandatu przez kandydata wystarczy największa liczba głosów oddanych na niego, tzw. \textit{zasada pierwszy na mecie}.

Powyżej opisany system jest kompromisem pomiędzy równymi prawami obywateli USA, a~równymi prawami poszczególnych stanów. W~ten sposób Izba Reprezentantów stara się reprezentować podobne wielkościowo grupy wyborców, podczas gdy w~Senacie każdy Stan ma taką samą liczbę senatorów.

Ponad dwieście lat funkcjonowania takiej formy ordynacji doprowadziło do zdominowania amerykańskiej sceny politycznej przez tylko dwie partie - obecnie Partię Demokratyczną i~Partię Republikańską. Należy dodać, że zaawansowany system prawyborów pozwala mieszkańcom na ingerencję w~wybór kandydatów na kandydatów danych partii, co pozwala na ingerencją lokalnej ludności w~decyzjach partyjnych.

\subsection{Zjednoczone Królestwo}
Brytyjski parlament składa się z~dwóch izb. Izba Lordów (ang. \textit{House of Lords}), czyli izba wyższa, ma skład niewybierany w~sposób demokratyczny, ale ukształtowany przez wielowiekową tradycję. Z~tego powodu nie organizuje się wyborów do niej.

Izba gmin (ang. \textit{House of Commons}), czyli izba niższa, jest odpowiednikiem amerykańskiej izby reprezentantów. Całe państwo jest podzielone 650 okręgów jednomandatowych. W~każdym okręgu kandydat zdobywający największą liczbę głosów zdobywa mandat.

Ordynacja w~Wielkiej Brytanii doprowadziła do dominacji dwóch partii - Partii Konserwatywnej i~Partii Pracy, ale w~przeciwieństwie do USA nie można mówić o dominacji całkowitej. W~Szkocji dominuje nacjonalistyczna Szkocka Partia Narodowa, a~w~Irlandii Północnej wygrywają lokalna partia unionistyczna \textit{Democratic Unionist Party} oraz pro-irlandzka republikańska partia \textit{Sinn Féin} \cite{Kwiatkowski}.

\subsection{Francja}
W 1958 roku premier Charles de Gaulle doprowadził do zmian konstytucyjnych we Francji powołując do ustanowienia V Republiki. Jedną z~wielu zmian była zamiana ordynacji wyborczej. Kraj podzielono na 577 okręgów jednomandatowych. Cechą charakterystyczną dla systemu francuskiego jest dwuturowość. Nie wystarczy, by kandydat uzyskał największą liczbę głosów w~swoim okręgu. Musi zdobyć 50\% wszystkich głosów. Jeżeli nie uzyska ich w~pierwszej turze, dwóch najlepszych kandydatów przechodzi do kolejnej tury, w~której głosowanie jest powtarzane. W~ten sposób de Gaulle chciał wyeliminować z~parlamentu ugrupowania skrajne. Taki system promuje ugrupowania najpopularniejsze i~dyskryminuje mniej popularne. Skutkiem tego większość prezydencka z~partią \textit{La République en marche} Emmanuela Macrona oraz z~ugrupowaniami satelickimi zdobywając 32,32\% w~I turze oraz 49,12\% w~drugiej turze ma aż 350 deputowanych do Zgromadzenia Narodowego, czyli 60,66\% wszystkich mandatów. Natomiast trzecia najbardziej popularna partia - nacjonalistyczny \textit{Rassemblement National} (dawniej \textit{Front National}) Marine le Pen zdobył 13,20\% głosów w~I turze oraz 8,75\% głosów w~II turze co przełożyło się na tylko 8 mandatów, czyli 1,39\% wszystkich dostępnych. Ta różnica dobrze pokazuje skalę nieproporcjonalności, co przez niektórych jest nazywane wypaczeniem demokracji\footnote{\url{https://www.gazetaprawna.pl/wiadomosci/artykuly/1051159,francja-ordynacja-wyborcza.html}} \cite{Myśliwiec}.

\subsection{Australia}
W Związku Australijskim jednomandatowe okręgi wyborcze są stosowane w~wyborach izby niższej tamtejszego parlamentu, czyli Izby Reprezentantów. Izba liczy 151 posłów, a~system jest podobny do jego brytyjskiego pierwowzoru. Cechą charakterystyczną jest natomiast tutaj preferencyjność. Głosowanie odbywa się poprzez uporządkowanie kandydatów od najbardziej preferowanego do najmniej, poprzez wpisanie kolejnych liczb przy nazwiskach kandydatów, przy czym wpisanie liczby 1 przy jednym kandydacie jest niezbędne, aby głos był ważny. Podczas liczenia głosów, najpierw sumuje się głosy oddane na kandydatów z~przypisanym numerem 1, a~następnie jeśli nikt nie uzyskał 50\% głosów eliminuje się najgorszego kandydata, a~jego głosy przypisuje się na kolejnych kandydatów według preferencji wyborcy (ci którzy dostali kolejne liczby 2, 3, itd.). Procedurę powtarza się do kiedy gdy któryś z~kandydatów uzyska 50\% głosów. Dzięki takiemu wyjściu głosy mniejszości mają większe znaczenie i~nie uznaje się je za zmarnowane \cite{Urbańczyk}.

\section{Propozycje wprowadzenia JOW w~Polsce}

Konkretne propozycje wprowadzenia Jednomandatowych Okręgów Wyborczych w~Polsce pojawiły się przy okazji wyborów prezydenckich w~2015 roku. Ich głównym postulatorem był Paweł Kukiz, jeden z~kandydatów na urząd Prezydenta RP\footnote{\url{https://www.newsweek.pl/polska/co-to-sa-jow-y-czy-jow-y-sa-potrzebne-pawel-kukiz/vvhstm8}}. Jego intencją było zmniejszenie wpływu władz partii politycznych na wybór kandydatów na posłów oraz konieczność reprezentowania swojej lokalnej społeczności. Mieszkaniec danego okręgu nie głosowałby na daną partię ale na danego konkretnego kandydata. Po zajęciu w~wyborach prezydenckich trzeciego miejsca z~wynikiem 20,80\%, założył ruch obywatelski Kukiz'15, którego głównym postulatem była zmiana systemu ordynacji wyborczej. Komitet Wyborczy Wyborców Kukiz'15 w~wyborach do Sejmu RP w~2015 roku uzyskał 8,81\%.

Na fali popularności postulatu wprowadzenia JOW Prezydent RP Bronisław Komorowski zaproponował przeprowadzenie referendum między innymi w~tej sprawie, które zostało przeprowadzone 17 czerwca 2015 roku. Jedno z~zadanych pytań brzmiało: \textit{Czy jest Pani/Pan za wprowadzeniem jednomandatowych okręgów wyborczych w~wyborach do Sejmu Rzeczypospolitej Polskiej?}. Pozytywna odpowiedź uzyskała 78,75\% głosów. Z~powodu frekwencji, która wyniosła jedynie 7,80\% i~nie przekroczyła konstytucyjnego progu połowy uprawnionych do głosowania, głosowanie zostało uznane za niewiążące\footnote{\url{https://referendum2015.pkw.gov.pl/}}.

\subsection{Konsekwencje}
Dotychczasowe propozycje wprowadzenia nowej ordynacji wyborczej w~Polsce proponowały przeprowadzenie reformy na wzór brytyjski, tzn. wg. zasady \textit{pierwszy na mecie zgarnia wszystko}. Państwo miałoby zostać podzielone na 460 okręgów - tyle ile miejsc w~Sejmie. Taka zmiana mogłaby mieć ogromne konsekwencje polityczne. Do tej pory wyborcy głosowali na partię, patrząc w~większej nie mierze na jej ogólnokrajowych liderów, natomiast na konkretnych kandydatów, przy których stawiali "X" \ na karcie do głosowania już nie zwracali tak dużej uwagi. Zazwyczaj głosuje się tylko na te partie, które mają szanse przekroczenia pięcioprocentowego progu wejścia do sejmu. Przy czym na przestrzeni ostatniej dekady doszło do polaryzacji i~większego znaczenia tylko dwóch partii - obozów politycznych - tj. Prawo i~Sprawiedliwość oraz Platforma (Koalicja) Obywatelska, które zdominowały polską scenę polityczną. Wprowadzenie zmian ordynacyjnych mogłoby doprowadzić do dalszej polaryzacji oraz jeszcze całkowitego zdominowania sceny politycznej przez te dwa obozy, podobnie jak to ma miejsce w~USA z~partiami Demokratyczną i~Republikańską.
Podejmując próby prognozy zachowania wyborców w~nowej ordynacji należałoby wziąć pod uwagę przejście głosów z~mniejszych partii na dwie największe. Już teraz w~wyborach do Senatu RP dochodziło do tzw. paktów senackich, polegających wystawieniu jednego wspólnego kandydata partii opozycyjnych wobec Prawa i~Sprawiedliwości\footnote{\url{https://wyborcza.pl/7,75398,25098998,wybory-2019-pakt-senacki-opozycji-przeciw-pis-porownujemy.html}}. W~wielu okręgach wyborca miał możliwość oddania głosów wyłącznie na jednego z~dwóch kandydatów. W~ten sposób elektoraty innych mniejszych partii były zmuszone do zagłosowania na jednego z~tych dwóch kandydatów. Podobna sytuacja mogłaby mieć miejsce przy wyborach do Sejmu z~zastosowaniem JOW i~należałoby uwzględnić ewentualne takie transfery głosów \cite{Wolnicki}.

\section{Wyniki ostatnich wyborów parlamentarnych w~Polsce}

Wybory parlamentarne w~2019 roku odbyły się w~dniu 13 października. Po 4 latach samodzielnej większości Prawa i~Sprawiedliwości w~obu izbach parlamentu, po raz kolejny udało się uzyskać tej partii większość parlamentarną, ale tylko w~Sejmie. Pogłębiające się podziały polityczne w~społeczeństwie doprowadziły do rekordowej mobilizacji elektoratu, co poskutkowało największą od 1989 roku frekwencją w~wyborach parlamentarnych, wynoszącą 61,74\%.

\subsection{Sejm RP}

W wyborach do Sejmu startowało 5 komitetów ogólnopolskich oraz 5 komitetów, które zarejestrowały listy tylko w~niektórych okręgach. Cechą szczególną tych wyborów było to, że komitety wyborcze (zarówno te koalicyjne jak i~samodzielne) reprezentowały przedstawicieli różnych partii politycznych, ale określających się jako tą samą stronę areny politycznej.

Komitety ogólnopolskie wraz z~głównymi ugrupowaniami na listach:
\begin{enumerate}
  \item Komitet Wyborczy Polskie Stronnictwo Ludowe - PSL, Kukiz'15,
  \item Komitet Wyborczy Prawo i~Sprawiedliwość - PiS, Solidarna Polska, Porozumienie,
  \item Komitet Wyborczy Sojusz Lewicy Demokratycznej - SLD, Lewica Razem, Wiosna,
  \item Komitet Wyborczy Konfederacja Wolność i~Niepodległość - Ruch Narodowy, KORWiN,
  \item Koalicyjny Komitet Wyborczy Koalicja Obywatelska PO .N IPL Zieloni.
\end{enumerate}


\begin{table}[h!]
\caption{Wyniki wyborów do Sejmu RP w~2019 roku}
\scalebox{0.9}{
\centering
\begin{tabular}{|l|r|l|r|l|}
\hline
\multicolumn{1}{|c|}{\textbf{Komitet}} &
  \multicolumn{1}{c|}{\textbf{Liczba głosów}} &
  \multicolumn{1}{c|}{\textbf{\% głosów}} &
  \multicolumn{1}{c|}{\textbf{Liczba mandatów}} &
  \multicolumn{1}{c|}{\textbf{\% mandatów}} \\ \hline
\textbf{PiS}                   & 8 051 935 & 43,59  & 235 & 51,09 \\ \hline
\textbf{KO}                    & 5 060 355 & 27,40  & 134 & 29,13 \\ \hline
\textbf{SLD}                   & 2 319 946 & 12,56  & 49  & 10,65 \\ \hline
\textbf{PSL}                   & 1 578 523 & 8,55   & 30  & 6,52  \\ \hline
\textbf{Konfederacja}          & 1 256 953 & 6,81   & 11  & 2,39  \\ \hline
\textbf{Mniejszość Niemiecka} & 32 094    & 0,17   & 1   & 0,22  \\ \hline
\textbf{Pozostałe komitety}  & 170 904   & 0,92   & 0   & 0     \\ \hline
\end{tabular}}
\caption*{Źródło: \url{sejmsenat2019.pkw.gov.pl}}
\end{table}

Z powyższej tabeli wynika, że z~obecnej ordynacji wyborczej najbardziej korzystają dwa największe ugrupowania, które dzięki przeliczaniu liczny oddanych głosów na liczbę mandatów wg. metody d'Hondta mają proporcjonalnie więcej mandatów niż odzwierciedlałoby to ich poparcie w~dniu wyborów. Najbardziej poszkodowany jest komitet Konfederacji Wolność i~Niepodległość, który przy 6,81\% głosów wprowadził tylko 11 posłów. Gdyby mandaty przydzielić w~sposób bezpośredni powinni ich mieć nawet 31. Warto zwrócić uwagę, że dzięki specjalnemu przywilejowi dla komitetów reprezentujących mniejszości narodowe\footnote{Dz.U.2020.0.1319 Art. 197. Zwolnienie komitetów wyborczych organizacji mniejszości narodowych z~warunku uzyskania co najmniej 5 \% ważnie oddanych głosów w~skali kraju} Mniejszość Niemiecka wprowadziła jednego posła. Ponadto 0,92\% społeczeństwo, które oddało głos na pozostałe komitety jest pozbawiona jakiejkolwiek reprezentacji sejmowej. Komitety znajdujące się zbiorczo w~tabeli pod nazwą \textit{Pozostałe komitety} są następujące: KWW Koalicja Bezpartyjni i~Samorządowcy, Skuteczni Piotra Liroya-Marca, Akcja Zawiedzionych Emerytów Rencistów oraz Prawica.

\subsection{Senat RP}

Od 2011 obowiązuje nowy kodeks wyborczy wprowadzający zmiany w~wyborach do Senatu oraz dzielący państwo na 100 jednomandatowych okręgów wyborczych\footnote{\url{https://prawo.gazetaprawna.pl/artykuly/535010,od-1-sierpnia-obowiazuje-nowy-kodeks-wyborczy-czyli-najblizsze-wybory-inaczej.html}}.

Z obawy przed zdobyciem większości przez PiS zarówno w~Sejmie jak i~w~Senacie, komitety głównych partii opozycyjnych (PSL, KO, SLD) wobec PiSu zdobyły się na tzw. pakt senacki. Wszystkie partie z~paktu popierały w~każdym okręgu tego samego kandydata. W~większości przypadków był to kandydat Platformy Obywatelskiej. Dzięki temu partie opozycyjne zdobyły łącznie 51 mandatów wobec 49 mandatów dla PiS, co oznacza, że mogą one stworzyć większość i~razem wybrać marszałka Senatu. Jest to też pierwszy przypadek w~historii III RP, kiedy główna formacja polityczna tworząca większość sejmową nie tworzy większości senackiej.

Wyniki wyborów według komitetów wyborczych:

\begin{table}[h!]
\caption{Wyniki wyborów do Senatu RP w~2019 roku}
\scalebox{0.9}{
\centering
\begin{tabular}{|l|r|l|r|}
\hline
\multicolumn{1}{|c|}{\textbf{Komitet}} &
  \multicolumn{1}{c|}{\textbf{Liczba głosów}} &
  \multicolumn{1}{c|}{\textbf{\% głosów}} &
  \multicolumn{1}{c|}{\textbf{Liczba mandatów}} \\ \hline
\textbf{KW Prawo i~Sprawiedliwość}                  & 8 110 193 & 44,56 & 48 \\ \hline
\textbf{KKW Koalicja Obywatelska PO .N iPL Zieloni} & 6 490 306 & 35,66 & 43 \\ \hline
\textbf{KW Polskie Stronnictwo Ludowe}              & 1 041 909 & 5,72  & 3  \\ \hline
\textbf{KW Sojusz Lewicy Demokratycznej}            & 415 745   & 2,28  & 2  \\ \hline
\textbf{KWW Lidia Staroń - Zawsze po stronie ludzi} & 106 035   & 0,58  & 1  \\ \hline
\textbf{KWW Krzysztofa Kwiatkowskiego}              & 79 348    & 0,44  & 1  \\ \hline
\textbf{KWW Wadim Tyszkiewicz}                      & 63 675    & 0,35  & 1  \\ \hline
\textbf{KWW Demokracja Obywatelska}                 & 44 956    & 0,25  & 1  \\ \hline
\textbf{Pozostałe}                                  & 1 849 181 & 10,16 & 0  \\ \hline
\end{tabular}}
\caption*{Źródło: \url{sejmsenat2019.pkw.gov.pl}}
\end{table}

Jak można zauważyć liczba głosów oddanych na KW Prawo i~Sprawiedliwość jest bardzo zbliżona do ich liczby w~wyborach do Sejmu. To znaczy, że zdecydowana większość wyborców PiS głosuje na kandydatów tego ugrupowania zarówno w~wyborach do Senatu jak i~do Sejmu, jednak poza nimi mało osób decyduje się na poparcie tej opcji. Koalicja Obywatelska jako główna partia opozycyjna, w~przeciwieństwie do pozostałych ugrupowań sejmowych ma znacznie lepszy wynik. Charakterystyczny jest większy udział mniejszych komitetów oraz kandydatów niezależnych. W~rzeczywistości, ci którzy zdobyli mandat senatorski, w~mniejszym lub większym stopniu współpracują z~najważniejszymi ugrupowaniami politycznymi.