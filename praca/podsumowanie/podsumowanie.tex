% !TeX encoding = UTF-8
% !TeX spellcheck = pl_PL

\def\filename{podsumowanie}

\chapter*{Podsumowanie}\label{ch:podsumowanie}

Głównym argumentem zwolenników wprowadzenia Jednomandatowych Okręgów Wyborczych w~Polsce jest założenie, że wyborcy w~takich okręgach będą mogli wybierać kandydatów, którzy będą lepiej reprezentować ich lokalną społeczność, będąc po prostu bliżej swoich wyborców. Jest to poniekąd prawda, ale należy pamiętać o~drugiej stronie medalu, którą możemy zauważyć w~państwach gdzie taki system już obowiązuje oraz w~wynikach niniejszego badania. Jest nim większa polaryzacja polityczna społeczeństwa oraz całkowita dominacja sceny politycznej przez tylko dwa ugrupowania polityczne.

Jak pokazują wyniki badania, nawet w~przypadku gdy wyborcy wezmą pod uwagę każdą opcję polityczną to i~tak tylko dwie są w~stanie wprowadzić swoich kandydatów do parlamentu. Warto jednak podkreślić, że ich przynależność partyjna nie będzie już tak bardzo implikować poglądów, ani głosowania po linii partii, gdyż bądź co bądź poseł w~pierwszej kolejności odpowiadałby przed wyborcami z~jego okręgu.

Mniejsze partie byłyby całkowicie zmarginalizowane i~pozbawione szans na wprowadzenie kogokolwiek do parlamentu, również biorąc pod uwagę, że bardzo często wyborca w~Polsce nie znajduje jakiegokolwiek kandydata, którego chce aby on go reprezentował. Zmaleć może frekwencja lub oddawanie tzw. zmarnowanych głosów na kandydatów niemających szans na wejście do Sejmu.

Należy pamiętać, że wszelkie wyniki w~badaniu ze względu na mocno niestabilną charakterystykę polskiej sceny politycznej są mocno hipotetyczne. Prawdopodobnie jednak wprowadzenie JOW byłoby zaczątkiem takiej stabilności, czyli znacznie mniejszą rotacją partii politycznych oraz być może również poszczególnych posłów. Pomimo naturalnych wad takiego modelu wynikających z~często nieracjonalnego zachowania zmienności decyzji politycznych wyborców, przedstawione badanie pozwala zobrazować jak potencjalna zmiana ordynacji mogłaby wpłynąć na polską politykę.

Powinno się jeszcze zaznaczyć, że przedstawione wyniki zostały skonstruowane na podstawie wyników wyborów parlamentarnych z~2019 roku oraz prezydenckich z~2020 roku. Od tej pory sytuacja polityczna zdążyła się zmienić, w~czym swój udział miały pandemia COVID-19 jak i~powstanie nowego ugrupowania Polska 2050 Szymona Hołowni.

