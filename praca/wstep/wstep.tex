% !TeX encoding = UTF-8
% !TeX spellcheck = pl_PL

\def\filename{Introduction}
\chapter*{Wstęp}

Państwa we współczesnym świecie zdominowane są przez ustroje demokratyczne, w~których obywatele wybierają swoich kandydatów do parlamentu w~różnych ordynacjach wyborczych. Wiele partii politycznych, w~tym także w~Polsce, postuluje reformę takiego systemu w~kierunku bardziej oddającym zróżnicowanie polityczne społeczeństwa, lub w~kierunku promującym większość polityczną w~danym społeczeństwie. Zauważa się trzy główne systemy ordynacji wyborczej bazujące na różnicach w~przyznawaniu liczby mandatów poselskich w~okręgu: wielomandatowe okręgi wyborcze (Sejm RP, Parlament Europejski), jednomandatowe okręgi wyborcze\footnote{W skrócie: JOW.} (USA, Wielka Brytania, Senat RP) oraz system mieszany, gdzie najczęściej część mandatów jest wybierana w~JOW, a~część z~listy wspólnej dla całego kraju będącego jednym dużym okręgiem wyborczym (Rosja, Niemcy, Węgry).

Niniejsza praca będzie stanowiła próbę stworzenia symulacji wprowadzenia JOW w~Polsce przy wyborach do niższej izby parlamentu RP - Sejmu na podstawie danych z~poprzednich wyborów parlamentarnych w~Polsce za pomocą narzędzi statystycznych oraz programu R Studio i~z~wykorzystaniem autorskiego kodu języka R. Celowi pracy będzie podporządkowana jej struktura: wstęp, następnie 3 rozdziały oraz zakończenie.

Pierwszy rozdział poruszy zagadnienie kontekstu politycznego odnośnie JOW w~Polsce, a~także wady i~zalety wprowadzania takiego systemu ordynacji. Różnice pomiędzy systemem jednomandatowym a~wielomandatowym będą zilustrowane przykładami ich implementacji w~różnych demokracjach na świecie. Zostaną również zaprezentowane tendencje polityczne Polaków na podstawie poprzednich wyborów oraz próba przewidzenia zachowania różnych ich grup po wprowadzeniu JOW, którego jednym z~możliwych skutków będzie dualizacja sceny politycznej.

Kolejny rozdział przedstawi specyfikację danych Państwowej Komisji Wyborczych, na podstawie której będą przeprowadzane symulacje. Ponieważ nowy system ordynacji będzie implikował nowy podział terytorialny okręgów wyborczych wraz ze~zwiększeniem ich liczby, zaprezentowany zostanie przykładowy podział przygotowany przez fundacje im. Madisona wraz z~możliwymi potencjalnymi manipulacjami w~tym zakresie. Scharakteryzowane będą również pakiety i~specyfikacje programu R użyte w~następnym rozdziale.

Rozdział trzeci będzie mieć już charakter praktyczny. Opisany zostanie program prognostyczny w~języku R przypisujący wyborców do nowych okręgów wyborczych wraz z~hipotetycznym nowym zachowaniem wynikającym z~przyjętych założeń i~określenia specyfiki nowego prawa wyborczego. Rozdział będzie również zawierać wizualizację uzyskanych wyników.

Opisane w~pracy narzędzie prognostyczne pozwoli zwizualizować następstwa zmiany systemu wyborczego w~Polsce. Wyniki będą mogły wnieść nowe statystyczne spojrzenie w~obecną dyskusję nad jednomandatowymi okręgami wyborczymi oraz będą stanowić pomoc dla politologów i~komentatorów politycznych jak i~partii politycznych biorących tę tematykę pod swoje rozważania, w~tym również te, które już postulują zmianę systemu elektoralnego.

%Ten rozdział zawiera propozycję struktury wstępu.
